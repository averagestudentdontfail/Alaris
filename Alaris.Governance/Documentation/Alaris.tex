% Options for packages loaded elsewhere
\PassOptionsToPackage{unicode}{hyperref}
\PassOptionsToPackage{hyphens}{url}
\PassOptionsToPackage{dvipsnames,svgnames,x11names}{xcolor}
%
\documentclass[
  11pt,
  a4paper]{article}

\usepackage{amsmath,amssymb}
\usepackage{iftex}
\ifPDFTeX
  \usepackage[T1]{fontenc}
  \usepackage[utf8]{inputenc}
  \usepackage{textcomp} % provide euro and other symbols
\else % if luatex or xetex
  \usepackage{unicode-math}
  \defaultfontfeatures{Scale=MatchLowercase}
  \defaultfontfeatures[\rmfamily]{Ligatures=TeX,Scale=1}
\fi
\usepackage{lmodern}
\ifPDFTeX\else  
    % xetex/luatex font selection
\fi
% Use upquote if available, for straight quotes in verbatim environments
\IfFileExists{upquote.sty}{\usepackage{upquote}}{}
\IfFileExists{microtype.sty}{% use microtype if available
  \usepackage[]{microtype}
  \UseMicrotypeSet[protrusion]{basicmath} % disable protrusion for tt fonts
}{}
\makeatletter
\@ifundefined{KOMAClassName}{% if non-KOMA class
  \IfFileExists{parskip.sty}{%
    \usepackage{parskip}
  }{% else
    \setlength{\parindent}{0pt}
    \setlength{\parskip}{6pt plus 2pt minus 1pt}}
}{% if KOMA class
  \KOMAoptions{parskip=half}}
\makeatother
\usepackage{xcolor}
\usepackage[margin=2.5cm]{geometry}
\setlength{\emergencystretch}{3em} % prevent overfull lines
\setcounter{secnumdepth}{-\maxdimen} % remove section numbering
% Make \paragraph and \subparagraph free-standing
\makeatletter
\ifx\paragraph\undefined\else
  \let\oldparagraph\paragraph
  \renewcommand{\paragraph}{
    \@ifstar
      \xxxParagraphStar
      \xxxParagraphNoStar
  }
  \newcommand{\xxxParagraphStar}[1]{\oldparagraph*{#1}\mbox{}}
  \newcommand{\xxxParagraphNoStar}[1]{\oldparagraph{#1}\mbox{}}
\fi
\ifx\subparagraph\undefined\else
  \let\oldsubparagraph\subparagraph
  \renewcommand{\subparagraph}{
    \@ifstar
      \xxxSubParagraphStar
      \xxxSubParagraphNoStar
  }
  \newcommand{\xxxSubParagraphStar}[1]{\oldsubparagraph*{#1}\mbox{}}
  \newcommand{\xxxSubParagraphNoStar}[1]{\oldsubparagraph{#1}\mbox{}}
\fi
\makeatother


\providecommand{\tightlist}{%
  \setlength{\itemsep}{0pt}\setlength{\parskip}{0pt}}\usepackage{longtable,booktabs,array}
\usepackage{calc} % for calculating minipage widths
% Correct order of tables after \paragraph or \subparagraph
\usepackage{etoolbox}
\makeatletter
\patchcmd\longtable{\par}{\if@noskipsec\mbox{}\fi\par}{}{}
\makeatother
% Allow footnotes in longtable head/foot
\IfFileExists{footnotehyper.sty}{\usepackage{footnotehyper}}{\usepackage{footnote}}
\makesavenoteenv{longtable}
\usepackage{graphicx}
\makeatletter
\newsavebox\pandoc@box
\newcommand*\pandocbounded[1]{% scales image to fit in text height/width
  \sbox\pandoc@box{#1}%
  \Gscale@div\@tempa{\textheight}{\dimexpr\ht\pandoc@box+\dp\pandoc@box\relax}%
  \Gscale@div\@tempb{\linewidth}{\wd\pandoc@box}%
  \ifdim\@tempb\p@<\@tempa\p@\let\@tempa\@tempb\fi% select the smaller of both
  \ifdim\@tempa\p@<\p@\scalebox{\@tempa}{\usebox\pandoc@box}%
  \else\usebox{\pandoc@box}%
  \fi%
}
% Set default figure placement to htbp
\def\fps@figure{htbp}
\makeatother
% definitions for citeproc citations
\NewDocumentCommand\citeproctext{}{}
\NewDocumentCommand\citeproc{mm}{%
  \begingroup\def\citeproctext{#2}\cite{#1}\endgroup}
\makeatletter
 % allow citations to break across lines
 \let\@cite@ofmt\@firstofone
 % avoid brackets around text for \cite:
 \def\@biblabel#1{}
 \def\@cite#1#2{{#1\if@tempswa , #2\fi}}
\makeatother
\newlength{\cslhangindent}
\setlength{\cslhangindent}{1.5em}
\newlength{\csllabelwidth}
\setlength{\csllabelwidth}{3em}
\newenvironment{CSLReferences}[2] % #1 hanging-indent, #2 entry-spacing
 {\begin{list}{}{%
  \setlength{\itemindent}{0pt}
  \setlength{\leftmargin}{0pt}
  \setlength{\parsep}{0pt}
  % turn on hanging indent if param 1 is 1
  \ifodd #1
   \setlength{\leftmargin}{\cslhangindent}
   \setlength{\itemindent}{-1\cslhangindent}
  \fi
  % set entry spacing
  \setlength{\itemsep}{#2\baselineskip}}}
 {\end{list}}
\usepackage{calc}
\newcommand{\CSLBlock}[1]{\hfill\break\parbox[t]{\linewidth}{\strut\ignorespaces#1\strut}}
\newcommand{\CSLLeftMargin}[1]{\parbox[t]{\csllabelwidth}{\strut#1\strut}}
\newcommand{\CSLRightInline}[1]{\parbox[t]{\linewidth - \csllabelwidth}{\strut#1\strut}}
\newcommand{\CSLIndent}[1]{\hspace{\cslhangindent}#1}

\usepackage{microtype}
\usepackage{booktabs}
\usepackage{amsmath}
\usepackage{amssymb}
\usepackage{mathtools}
\usepackage{tikz}
\usepackage{pgfplots}
\pgfplotsset{compat=1.18}

% Mathematical operators (only if not already defined)
\providecommand{\E}{\mathbb{E}}
\providecommand{\Var}{\operatorname{Var}}
\providecommand{\Cov}{\operatorname{Cov}}
\providecommand{\Corr}{\operatorname{Corr}}
\providecommand{\sgn}{\operatorname{sgn}}

% Indicator function
\providecommand{\ind}[1]{\mathbf{1}_{#1}}

% Common sets
\providecommand{\R}{\mathbb{R}}
\providecommand{\N}{\mathbb{N}}
\providecommand{\Z}{\mathbb{Z}}

% Probability spaces
\providecommand{\F}{\mathcal{F}}
\providecommand{\Q}{\mathbb{Q}}
\providecommand{\Prob}{\mathbb{P}}
\makeatletter
\@ifpackageloaded{tcolorbox}{}{\usepackage[skins,breakable]{tcolorbox}}
\@ifpackageloaded{fontawesome5}{}{\usepackage{fontawesome5}}
\definecolor{quarto-callout-color}{HTML}{909090}
\definecolor{quarto-callout-note-color}{HTML}{0758E5}
\definecolor{quarto-callout-important-color}{HTML}{CC1914}
\definecolor{quarto-callout-warning-color}{HTML}{EB9113}
\definecolor{quarto-callout-tip-color}{HTML}{00A047}
\definecolor{quarto-callout-caution-color}{HTML}{FC5300}
\definecolor{quarto-callout-color-frame}{HTML}{acacac}
\definecolor{quarto-callout-note-color-frame}{HTML}{4582ec}
\definecolor{quarto-callout-important-color-frame}{HTML}{d9534f}
\definecolor{quarto-callout-warning-color-frame}{HTML}{f0ad4e}
\definecolor{quarto-callout-tip-color-frame}{HTML}{02b875}
\definecolor{quarto-callout-caution-color-frame}{HTML}{fd7e14}
\makeatother
\makeatletter
\@ifpackageloaded{caption}{}{\usepackage{caption}}
\AtBeginDocument{%
\ifdefined\contentsname
  \renewcommand*\contentsname{Table of contents}
\else
  \newcommand\contentsname{Table of contents}
\fi
\ifdefined\listfigurename
  \renewcommand*\listfigurename{List of Figures}
\else
  \newcommand\listfigurename{List of Figures}
\fi
\ifdefined\listtablename
  \renewcommand*\listtablename{List of Tables}
\else
  \newcommand\listtablename{List of Tables}
\fi
\ifdefined\figurename
  \renewcommand*\figurename{Figure}
\else
  \newcommand\figurename{Figure}
\fi
\ifdefined\tablename
  \renewcommand*\tablename{Table}
\else
  \newcommand\tablename{Table}
\fi
}
\@ifpackageloaded{float}{}{\usepackage{float}}
\floatstyle{ruled}
\@ifundefined{c@chapter}{\newfloat{codelisting}{h}{lop}}{\newfloat{codelisting}{h}{lop}[chapter]}
\floatname{codelisting}{Listing}
\newcommand*\listoflistings{\listof{codelisting}{List of Listings}}
\usepackage{amsthm}
\theoremstyle{definition}
\newtheorem{definition}{Definition}[section]
\theoremstyle{plain}
\newtheorem{theorem}{Theorem}[section]
\theoremstyle{remark}
\AtBeginDocument{\renewcommand*{\proofname}{Proof}}
\newtheorem*{remark}{Remark}
\newtheorem*{solution}{Solution}
\newtheorem{refremark}{Remark}[section]
\newtheorem{refsolution}{Solution}[section]
\makeatother
\makeatletter
\makeatother
\makeatletter
\@ifpackageloaded{caption}{}{\usepackage{caption}}
\@ifpackageloaded{subcaption}{}{\usepackage{subcaption}}
\makeatother

\ifLuaTeX
\usepackage[bidi=basic]{babel}
\else
\usepackage[bidi=default]{babel}
\fi
\babelprovide[main,import]{australian}
% get rid of language-specific shorthands (see #6817):
\let\LanguageShortHands\languageshorthands
\def\languageshorthands#1{}
\ifLuaTeX
  \usepackage[english]{selnolig} % disable illegal ligatures
\fi
\usepackage{bookmark}

\IfFileExists{xurl.sty}{\usepackage{xurl}}{} % add URL line breaks if available
\urlstyle{same} % disable monospaced font for URLs
\hypersetup{
  pdftitle={Alaris},
  pdfauthor={Kiran Nath},
  pdflang={en-AU},
  pdfkeywords={American Options, Spectral Methods, Free Boundary
Problems, Negative Interest Rates, Optimal Stopping, Earnings
Volatility, Calendar Spreads, Fault Detection},
  colorlinks=true,
  linkcolor={blue},
  filecolor={Maroon},
  citecolor={Blue},
  urlcolor={Blue},
  pdfcreator={LaTeX via pandoc}}


\title{Alaris}
\usepackage{etoolbox}
\makeatletter
\providecommand{\subtitle}[1]{% add subtitle to \maketitle
  \apptocmd{\@title}{\par {\large #1 \par}}{}{}
}
\makeatother
\subtitle{Earnings Volatility Calendar Spread System}
\author{Kiran Nath}
\date{2025-12-01}

\begin{document}
\maketitle
\begin{abstract}
This document provides a complete mathematical specification of the
Alaris quantitative trading system. Beginning from first principles, we
derive the option pricing methodology for American-style securities
under both positive and negative interest rate regimes, formalise the
earnings-based volatility trading strategy with mathematically precise
entry criteria, and specify the complete fault monitoring framework as a
system of inequalities and logical predicates. Every decision rule in
the system corresponds to a mathematically specified condition, ensuring
deterministic behaviour. Where stochasticity is inherent to the
underlying processes, we explicitly characterise the probabilistic
assumptions and derive appropriate estimators with known statistical
properties.
\end{abstract}


\section{Introduction}\label{introduction}

\subsection{Purpose and Scope}\label{purpose-and-scope}

The Alaris system implements a systematic approach to capturing the
implied volatility premium that manifests around corporate earnings
announcements. This document serves three purposes: to provide a
rigorous mathematical foundation for all computational methods employed,
to specify formally the decision rules governing strategy execution, and
to derive the complete set of fault conditions that the system monitors.

A distinguishing characteristic of this specification is its emphasis on
mathematical completeness. Every algorithmic decision in Alaris
corresponds to an evaluable predicate derived from the mathematical
framework. Where the underlying phenomena are inherently stochastic, we
explicitly state the probabilistic model assumptions and derive
estimators with known convergence properties.

\subsection{Document Organisation}\label{document-organisation}

The document proceeds as follows. Section 2 establishes the mathematical
framework for option pricing, deriving the free boundary formulation and
its integral equation transformation. Section 3 extends this framework
to the double boundary regime arising under negative interest rates.
Section 4 develops the volatility trading strategy, specifying the
signal generation predicates with their academic provenance. Section 5
formalises the complete fault detection and monitoring system as a set
of mathematical inequalities. Section 6 addresses the inherent
stochasticity in certain components and specifies how this is accounted
for.

\section{The Option Pricing
Framework}\label{the-option-pricing-framework}

\subsection{Stochastic Foundation}\label{stochastic-foundation}

\subsubsection{The Risk-Neutral Measure}\label{the-risk-neutral-measure}

The Alaris pricing methodology operates under the risk-neutral
probability measure \(\mathbb{Q}\). Under this measure, the asset price
process \(S(t)\) satisfies the stochastic differential equation:

\begin{equation}\phantomsection\label{eq-sde}{
\frac{dS(t)}{S(t)} = (r - q) \, dt + \sigma \, dW^{\mathbb{Q}}(t)
}\end{equation}

where:

\begin{itemize}
\tightlist
\item
  \(r \in \mathbb{R}\) denotes the continuously compounded risk-free
  interest rate
\item
  \(q \in \mathbb{R}\) denotes the continuous dividend yield\\
\item
  \(\sigma > 0\) denotes the instantaneous volatility
\item
  \(W^{\mathbb{Q}}(t)\) is a standard Brownian motion under
  \(\mathbb{Q}\)
\end{itemize}

\begin{tcolorbox}[enhanced jigsaw, breakable, rightrule=.15mm, toptitle=1mm, leftrule=.75mm, left=2mm, opacitybacktitle=0.6, colbacktitle=quarto-callout-note-color!10!white, colframe=quarto-callout-note-color-frame, colback=white, coltitle=black, opacityback=0, titlerule=0mm, toprule=.15mm, bottomrule=.15mm, title=\textcolor{quarto-callout-note-color}{\faInfo}\hspace{0.5em}{Parameter Domain Extension}, arc=.35mm, bottomtitle=1mm]

Unlike classical treatments that assume \(r \geq 0\), Alaris explicitly
permits \(r < 0\) to accommodate negative interest rate environments.
This extension requires careful treatment of the exercise region
topology, addressed in Section 3.

\end{tcolorbox}

The solution to Equation~\ref{eq-sde} is the geometric Brownian motion:

\begin{equation}\phantomsection\label{eq-gbm-solution}{
S(t) = S(0) \exp\left[\left(r - q - \frac{\sigma^2}{2}\right)t + \sigma W^{\mathbb{Q}}(t)\right]
}\end{equation}

\subsubsection{Filtration and Information
Structure}\label{filtration-and-information-structure}

Let \((\Omega, \mathcal{F}, \mathbb{Q})\) be a complete probability
space equipped with the natural filtration
\(\{\mathcal{F}_t\}_{t \geq 0}\) generated by the Brownian motion
\(W^{\mathbb{Q}}(t)\), augmented by null sets. All stopping times
referenced in this document are with respect to this filtration.

\subsection{The Free Boundary Problem}\label{the-free-boundary-problem}

\subsubsection{Value Function
Characterisation}\label{value-function-characterisation}

For an American put option with strike \(K\) and maturity \(T\), define
the value function
\(V: [0,T] \times \mathbb{R}_{>0} \to \mathbb{R}_{\geq 0}\) as:

\begin{equation}\phantomsection\label{eq-value-function}{
V(t, s) = \sup_{\tau \in \mathcal{T}_{t,T}} \mathbb{E}^{\mathbb{Q}}\left[e^{-r(\tau - t)}(K - S(\tau))^+ \,\Big|\, S(t) = s\right]
}\end{equation}

where \(\mathcal{T}_{t,T}\) denotes the set of stopping times taking
values in \([t, T]\).

The supremum in Equation~\ref{eq-value-function} is attained by the
optimal stopping time:

\begin{equation}\phantomsection\label{eq-optimal-stopping}{
\tau^* = \inf\{u \in [t, T] : S(u) \leq B(T - u)\}
}\end{equation}

where \(B(\cdot)\) is the optimal exercise boundary function.

\subsubsection{The Free Boundary}\label{the-free-boundary}

\begin{definition}[Exercise
Boundary]\protect\hypertarget{def-exercise-boundary}{}\label{def-exercise-boundary}

The optimal exercise boundary \(B: [0, T] \to \mathbb{R}_{>0}\) is the
unique continuous function satisfying:

\begin{enumerate}
\def\labelenumi{\arabic{enumi}.}
\tightlist
\item
  \textbf{Value Matching}: \(V(t, B(T-t)) = K - B(T-t)\) for all
  \(t \in [0, T]\)
\item
  \textbf{Smooth Pasting}:
  \(\frac{\partial V}{\partial s}(t, B(T-t)) = -1\) for all
  \(t \in (0, T)\)
\end{enumerate}

\end{definition}

The value function satisfies the variational inequality:

\begin{equation}\phantomsection\label{eq-variational}{
\max\left\{\mathcal{L}V - rV, \; (K - s) - V\right\} = 0
}\end{equation}

where the infinitesimal generator is:

\[
\mathcal{L}V = \frac{\partial V}{\partial t} + (r-q)s\frac{\partial V}{\partial s} + \frac{1}{2}\sigma^2 s^2 \frac{\partial^2 V}{\partial s^2}
\]

\subsection{The Integral Equation
Formulation}\label{the-integral-equation-formulation}

\subsubsection{Early Exercise Premium
Decomposition}\label{early-exercise-premium-decomposition}

The American option value decomposes into European option value plus
early exercise premium:

\begin{equation}\phantomsection\label{eq-eep-decomposition}{
V(\tau, s) = v(\tau, s) + \mathcal{P}(\tau, s)
}\end{equation}

where \(\tau = T - t\) denotes time to maturity, \(v(\tau, s)\) is the
European put price, and \(\mathcal{P}(\tau, s)\) is the early exercise
premium.

\begin{theorem}[Kim Integral
Representation]\protect\hypertarget{thm-kim-integral}{}\label{thm-kim-integral}

The early exercise premium satisfies:

\begin{equation}\phantomsection\label{eq-kim-integral}{
\mathcal{P}(\tau, s) = \int_0^{\tau} e^{-ru} \left[rK \Phi(-d_-(u, s/B(u))) - qse^{(r-q)u}\Phi(-d_+(u, s/B(u)))\right] du
}\end{equation}

where:

\[
d_{\pm}(u, z) = \frac{\ln(z) + (r-q)u \pm \frac{1}{2}\sigma^2 u}{\sigma\sqrt{u}}
\]

and \(\Phi(\cdot)\) denotes the cumulative standard normal distribution
function.

\end{theorem}

\subsubsection{The Boundary Integral
Equation}\label{the-boundary-integral-equation}

Applying the value-matching condition at the boundary yields the
nonlinear integral equation for \(B(\tau)\):

\begin{equation}\phantomsection\label{eq-boundary-integral}{
K - B(\tau) = v(\tau, B(\tau)) + \int_0^{\tau} \mathcal{K}(\tau, u, B(\tau), B(u)) \, du
}\end{equation}

where the kernel \(\mathcal{K}\) is:

\[
\mathcal{K}(\tau, u, b, b') = e^{-ru}\left[rK\Phi(-d_-(u, b/b')) - qb \cdot e^{(r-q)u}\Phi(-d_+(u, b/b'))\right]
\]

\subsubsection{Asymptotic Behaviour}\label{asymptotic-behaviour}

The boundary function exhibits the following asymptotic behaviour:

\textbf{Near Expiration} (\(\tau \to 0^+\)):
\begin{equation}\phantomsection\label{eq-near-expiry}{
\lim_{\tau \to 0^+} B(\tau) = \begin{cases}
K & \text{if } r \geq q \\
K \cdot \frac{r}{q} & \text{if } r < q \text{ and } r > 0
\end{cases}
}\end{equation}

\textbf{Long Maturity} (\(\tau \to \infty\)):
\begin{equation}\phantomsection\label{eq-perpetual-boundary}{
\lim_{\tau \to \infty} B(\tau) = B_{\infty} = K \cdot \frac{\lambda_-}{\lambda_- - 1}
}\end{equation}

where \(\lambda_-\) is the negative root of the characteristic equation:

\begin{equation}\phantomsection\label{eq-characteristic}{
\frac{1}{2}\sigma^2\lambda^2 + \left(r - q - \frac{\sigma^2}{2}\right)\lambda - r = 0
}\end{equation}

\subsection{The QD+ Approximation
Method}\label{the-qd-approximation-method}

\subsubsection{Quasi-Analytic Framework}\label{quasi-analytic-framework}

The Alaris system employs the QD+ approximation method of Andersen,
Lake, and Offengenden \citeproc{ref-andersen2016}{(2016)} as the first
stage of boundary estimation. This method provides an accurate
quasi-analytic approximation that serves as initialisation for
subsequent refinement.

\begin{definition}[QD+ Boundary
Approximation]\protect\hypertarget{def-qd-approximation}{}\label{def-qd-approximation}

The QD+ approximation expresses the boundary as:

\begin{equation}\phantomsection\label{eq-qd-boundary}{
B^{\text{QD}}(\tau) = K \cdot \frac{\lambda(\tau)}{\lambda(\tau) - 1}
}\end{equation}

where \(\lambda(\tau)\) is obtained from a modified characteristic
equation with time-dependent correction:

\[
\frac{1}{2}\sigma^2\lambda^2 + \left(r - q - \frac{\sigma^2}{2}\right)\lambda - r \cdot h(\tau) = 0
\]

with \(h(\tau) = 1 - e^{-r\tau}\).

\end{definition}

\subsubsection{Super Halley's Method}\label{super-halleys-method}

The QD+ boundary equation is solved using Super Halley's method, a
third-order root-finding algorithm:

\begin{equation}\phantomsection\label{eq-super-halley}{
S_{n+1} = S_n - \frac{f(S_n)}{f'(S_n)} \cdot \frac{1}{1 - \frac{f(S_n) \cdot f''(S_n)}{2[f'(S_n)]^2}}
}\end{equation}

\begin{theorem}[Super Halley
Convergence]\protect\hypertarget{thm-halley-convergence}{}\label{thm-halley-convergence}

For a function \(f\) with simple root \(S^*\) and continuous third
derivative in a neighbourhood of \(S^*\), Super Halley's method
converges cubically:

\[
|S_{n+1} - S^*| = O(|S_n - S^*|^3)
\]

\end{theorem}

The implementation specifies convergence tolerance
\(\epsilon = 10^{-10}\) and maximum iterations \(N_{\max} = 50\).

\section{Double Boundary Regime Under Negative
Rates}\label{double-boundary-regime-under-negative-rates}

\subsection{Regime Characterisation}\label{regime-characterisation}

\subsubsection{Mathematical Conditions for Double
Boundaries}\label{mathematical-conditions-for-double-boundaries}

The emergence of double boundaries occurs under specific parameter
configurations involving negative interest rates.

\begin{definition}[Double Boundary
Regime]\protect\hypertarget{def-double-boundary-regime}{}\label{def-double-boundary-regime}

The double boundary regime is characterised by the predicate:

\begin{equation}\phantomsection\label{eq-double-regime}{
\mathcal{R}_{\text{double}} \equiv (q < r < 0)
}\end{equation}

When \(\mathcal{R}_{\text{double}}\) holds, the optimal exercise region
for an American put is the set:

\[
\mathcal{E}(\tau) = \{s > 0 : Y(\tau) \leq s \leq B(\tau)\}
\]

where \(B(\tau)\) is the upper boundary and \(Y(\tau)\) is the lower
boundary.

\end{definition}

\subsubsection{Physical Interpretation}\label{physical-interpretation}

In the double boundary regime, immediate exercise is optimal only when
the asset price falls within a bounded interval. This arises because:

\begin{enumerate}
\def\labelenumi{\arabic{enumi}.}
\tightlist
\item
  When \(s > B(\tau)\): Continuation is optimal due to potential for
  deeper in-the-money scenarios
\item
  When \(Y(\tau) \leq s \leq B(\tau)\): Immediate exercise is optimal
\item
  When \(s < Y(\tau)\): Continuation is optimal because the negative
  dividend yield implies expected price appreciation
\end{enumerate}

\subsubsection{Critical Volatility
Threshold}\label{critical-volatility-threshold}

The behaviour of the double boundary system depends on whether
volatility exceeds a critical threshold.

\begin{definition}[Critical
Volatility]\protect\hypertarget{def-critical-volatility}{}\label{def-critical-volatility}

Define the critical volatility as:

\begin{equation}\phantomsection\label{eq-critical-vol}{
\sigma^* = \sqrt{\frac{2|r|(r-q)}{q}}
}\end{equation}

When \(\sigma > \sigma^*\), the boundaries intersect at finite time
\(\tau^*\).

\end{definition}

\subsection{Modified Integral Equation
System}\label{modified-integral-equation-system}

\subsubsection{Two-Boundary Integral
Representation}\label{two-boundary-integral-representation}

Under the double boundary regime, the American put value satisfies:

\begin{equation}\phantomsection\label{eq-double-eep}{
V(\tau, s) = v(\tau, s) + \mathcal{P}_B(\tau, s) - \mathcal{P}_Y(\tau, s)
}\end{equation}

where:

\[
\mathcal{P}_B(\tau, s) = \int_0^{\tau} e^{-ru}\left[rK\Phi(-d_-(u, s/B)) - qse^{(r-q)u}\Phi(-d_+(u, s/B))\right]du
\]

\[
\mathcal{P}_Y(\tau, s) = \int_0^{\tau} e^{-ru}\left[rK\Phi(-d_-(u, s/Y)) - qse^{(r-q)u}\Phi(-d_+(u, s/Y))\right]du
\]

\subsubsection{Decoupled Fixed-Point
Systems}\label{decoupled-fixed-point-systems}

A key mathematical result enabling efficient computation is the
decoupling of the boundary equations.

\begin{theorem}[Boundary
Decoupling]\protect\hypertarget{thm-decoupling}{}\label{thm-decoupling}

The upper boundary \(B(\tau)\) can be computed independently of
\(Y(\tau)\) using:

\begin{equation}\phantomsection\label{eq-upper-fixed}{
K - B(\tau) = v(\tau, B(\tau)) + \int_0^{\tau} \mathcal{K}_B(\tau, u, B(\tau), B(u)) \, du
}\end{equation}

Given \(B(\tau)\), the lower boundary \(Y(\tau)\) is then determined by:

\begin{equation}\phantomsection\label{eq-lower-fixed}{
K - Y(\tau) = v(\tau, Y(\tau)) + \int_0^{\tau} \mathcal{K}_B(\tau, u, Y(\tau), B(u)) - \mathcal{K}_Y(\tau, u, Y(\tau), Y(u)) \, du
}\end{equation}

\end{theorem}

\subsubsection{FP-B' Stabilisation}\label{fp-b-stabilisation}

The Alaris system employs the FP-B' stabilised iteration scheme from
Healy \citeproc{ref-healy2021}{(2021)} to prevent oscillatory behaviour
in long-maturity options.

\begin{definition}[FP-B'
Iteration]\protect\hypertarget{def-fpb-prime}{}\label{def-fpb-prime}

At iteration \(n\), the stabilised update for the lower boundary is:

\[
Y^{(n+1)}(\tau) = \mathcal{T}_Y\left[Y^{(n)}; B^{(n+1)}\right]
\]

where the operator \(\mathcal{T}_Y\) incorporates the just-computed
upper boundary \(B^{(n+1)}\) rather than the previous iteration
\(B^{(n)}\).

\end{definition}

This stabilisation ensures convergence for maturities up to \(T = 15\)
years under the Healy benchmark parameters.

\subsection{Boundary Validation
Predicates}\label{boundary-validation-predicates}

The computed boundaries must satisfy mathematical constraints from the
free boundary theory.

\begin{definition}[Boundary Validity
Conditions]\protect\hypertarget{def-boundary-validation}{}\label{def-boundary-validation}

A computed boundary pair \((B(\tau), Y(\tau))\) is valid if and only if:

\begin{align}
\mathcal{V}_1 &\equiv B(\tau) > 0 \quad \forall \tau \in [0, T] \tag{Positivity - Upper} \\
\mathcal{V}_2 &\equiv Y(\tau) > 0 \quad \forall \tau \in [0, T] \tag{Positivity - Lower} \\
\mathcal{V}_3 &\equiv B(\tau) > Y(\tau) \quad \forall \tau \in [0, T] \tag{Ordering} \\
\mathcal{V}_4 &\equiv B(\tau) < K \quad \forall \tau \in [0, T] \tag{Put Constraint - Upper} \\
\mathcal{V}_5 &\equiv Y(\tau) < K \quad \forall \tau \in [0, T] \tag{Put Constraint - Lower}
\end{align}

The aggregate validity predicate is:

\[
\mathcal{V}_{\text{boundary}} \equiv \bigwedge_{i=1}^{5} \mathcal{V}_i
\]

\end{definition}

\section{The Earnings Volatility Trading
Strategy}\label{the-earnings-volatility-trading-strategy}

\subsection{Theoretical Foundation}\label{theoretical-foundation}

\subsubsection{The Volatility Risk
Premium}\label{the-volatility-risk-premium}

The earnings volatility strategy exploits the documented phenomenon that
implied volatility systematically exceeds realised volatility around
corporate earnings announcements. Let:

\begin{itemize}
\tightlist
\item
  \(\sigma_I(t)\) denote the market-implied volatility at time \(t\)
\item
  \(\sigma_R(t)\) denote the realised volatility over the same horizon
\end{itemize}

The volatility risk premium is:

\begin{equation}\phantomsection\label{eq-vrp}{
\text{VRP}(t) = \sigma_I(t) - \sigma_R(t)
}\end{equation}

\phantomsection\label{assumption-vrp}
\subsection{Earnings VRP Assumption}\label{earnings-vrp-assumption}

For securities with scheduled earnings announcements at time \(T_E\),
the volatility risk premium exhibits the pattern:

\[
\mathbb{E}[\text{VRP}(t)] > 0 \quad \text{for } t \in [T_E - \Delta, T_E)
\]

where \(\Delta \approx 5\text{-}7\) trading days, with mean reversion
following the announcement.

\subsection{Signal Generation
Predicates}\label{signal-generation-predicates}

\subsubsection{The Atilgan Criteria}\label{the-atilgan-criteria}

The signal generation system implements the criteria from Atilgan et al.
\citeproc{ref-atilgan2014}{(2014)}, expressed as formal predicates.

\begin{definition}[Trading Signal
Predicates]\protect\hypertarget{def-signal-predicates}{}\label{def-signal-predicates}

For a security with symbol \(\xi\) evaluated at time \(t\) with earnings
date \(T_E\), define:

\textbf{IV/RV Ratio Criterion}:
\begin{equation}\phantomsection\label{eq-ivrv-criterion}{
\mathcal{S}_1(\xi, t) \equiv \frac{\sigma_I^{30}(\xi, t)}{\sigma_R^{30}(\xi, t)} \geq 1.25
}\end{equation}

\textbf{Term Structure Criterion}:
\begin{equation}\phantomsection\label{eq-term-criterion}{
\mathcal{S}_2(\xi, t) \equiv \nabla_\tau \sigma_I(\xi, t) \leq -0.00406
}\end{equation}

where \(\nabla_\tau \sigma_I\) denotes the slope of the implied
volatility term structure.

\textbf{Liquidity Criterion}:
\begin{equation}\phantomsection\label{eq-volume-criterion}{
\mathcal{S}_3(\xi, t) \equiv \bar{V}^{30}(\xi, t) \geq 1{,}500{,}000
}\end{equation}

where \(\bar{V}^{30}\) is the 30-day average daily trading volume.

\end{definition}

\subsubsection{Signal Strength
Classification}\label{signal-strength-classification}

\begin{definition}[Signal Strength
Function]\protect\hypertarget{def-signal-strength}{}\label{def-signal-strength}

The signal strength function
\(\Sigma: \{\text{true}, \text{false}\}^3 \to \{\text{Recommended}, \text{Consider}, \text{Avoid}\}\)
is:

\begin{equation}\phantomsection\label{eq-signal-strength}{
\Sigma(\mathcal{S}_1, \mathcal{S}_2, \mathcal{S}_3) = \begin{cases}
\text{Recommended} & \text{if } \mathcal{S}_1 \land \mathcal{S}_2 \land \mathcal{S}_3 \\
\text{Consider} & \text{if exactly two of } \{\mathcal{S}_1, \mathcal{S}_2, \mathcal{S}_3\} \text{ hold} \\
\text{Avoid} & \text{otherwise}
\end{cases}
}\end{equation}

\end{definition}

\subsection{Realised Volatility
Estimation}\label{realised-volatility-estimation}

\subsubsection{The Yang-Zhang Estimator}\label{the-yang-zhang-estimator}

The Alaris system employs the Yang and Zhang
\citeproc{ref-yang2000}{(2000)} estimator for realised volatility, which
is efficient for OHLC (open-high-low-close) data and robust to opening
gaps.

\begin{definition}[Yang-Zhang
Estimator]\protect\hypertarget{def-yang-zhang}{}\label{def-yang-zhang}

Given \(n\) trading days with price data \((O_i, H_i, L_i, C_i)\),
define:

\textbf{Overnight Variance}: \[
\sigma_o^2 = \frac{1}{n-1} \sum_{i=1}^{n} \left(\ln\frac{O_i}{C_{i-1}} - \bar{o}\right)^2
\]

\textbf{Open-to-Close Variance}: \[
\sigma_c^2 = \frac{1}{n-1} \sum_{i=1}^{n} \left(\ln\frac{C_i}{O_i} - \bar{c}\right)^2
\]

\textbf{Rogers-Satchell Variance}: \[
\sigma_{RS}^2 = \frac{1}{n} \sum_{i=1}^{n} \left[\ln\frac{H_i}{C_i} \cdot \ln\frac{H_i}{O_i} + \ln\frac{L_i}{C_i} \cdot \ln\frac{L_i}{O_i}\right]
\]

\textbf{Combined Estimator}:
\begin{equation}\phantomsection\label{eq-yang-zhang}{
\sigma_{YZ}^2 = \sigma_o^2 + k\sigma_c^2 + (1-k)\sigma_{RS}^2
}\end{equation}

where \(k = 0.34 / (1.34 + \frac{n+1}{n-1})\).

\end{definition}

\begin{theorem}[Yang-Zhang
Efficiency]\protect\hypertarget{thm-yz-efficiency}{}\label{thm-yz-efficiency}

The Yang-Zhang estimator achieves efficiency factor \(\eta \approx 8\)
relative to close-to-close volatility estimation, meaning it requires
approximately \(1/8\) as many observations for equivalent precision.

\end{theorem}

\subsection{Term Structure Analysis}\label{term-structure-analysis}

\subsubsection{Linear Term Structure
Model}\label{linear-term-structure-model}

The implied volatility term structure is modelled as:

\begin{equation}\phantomsection\label{eq-term-structure}{
\sigma_I(\tau) = \alpha + \beta \cdot \tau + \epsilon(\tau)
}\end{equation}

where \(\tau\) is days to expiration.

The slope parameter \(\beta\) is estimated via ordinary least squares:

\begin{equation}\phantomsection\label{eq-ols-slope}{
\hat{\beta} = \frac{\sum_{i=1}^{m}(\tau_i - \bar{\tau})(\sigma_i - \bar{\sigma})}{\sum_{i=1}^{m}(\tau_i - \bar{\tau})^2}
}\end{equation}

\begin{definition}[Term Structure Trading
Signal]\protect\hypertarget{def-term-structure-signal}{}\label{def-term-structure-signal}

A negative slope \(\hat{\beta} < \beta^* = -0.00406\) indicates elevated
short-dated implied volatility relative to longer tenors, consistent
with the pre-earnings IV premium.

\end{definition}

\subsection{Position Sizing: The Kelly
Criterion}\label{position-sizing-the-kelly-criterion}

\subsubsection{Optimal Growth Framework}\label{optimal-growth-framework}

Position sizing employs the Kelly criterion to maximise long-run
geometric growth rate.

\begin{theorem}[Kelly
Criterion]\protect\hypertarget{thm-kelly}{}\label{thm-kelly}

For a bet with probability \(p\) of winning return \(b\) and probability
\((1-p)\) of losing stake, the growth-optimal fraction is:

\begin{equation}\phantomsection\label{eq-kelly-full}{
f^* = \frac{p(b+1) - 1}{b} = \frac{pb - (1-p)}{b}
}\end{equation}

For small expected returns, this simplifies to:

\[
f^* \approx \frac{\mu}{\sigma^2}
\]

where \(\mu\) is expected return and \(\sigma^2\) is variance.

\end{theorem}

\subsubsection{Fractional Kelly
Implementation}\label{fractional-kelly-implementation}

The Alaris system employs fractional Kelly sizing to reduce variance:

\begin{equation}\phantomsection\label{eq-fractional-kelly}{
f_{\text{actual}} = \kappa \cdot f^* \quad \text{with } \kappa \in \{0.01, 0.02\}
}\end{equation}

The fraction \(\kappa\) depends on signal strength:

\[
\kappa = \begin{cases}
0.02 & \text{if } \Sigma = \text{Recommended} \\
0.01 & \text{if } \Sigma = \text{Consider} \\
0 & \text{if } \Sigma = \text{Avoid}
\end{cases}
\]

\section{Fault Detection and Monitoring
System}\label{fault-detection-and-monitoring-system}

\subsection{Overview of the Fault
Framework}\label{overview-of-the-fault-framework}

The Alaris fault detection system is designed to identify conditions
where model assumptions may be violated or execution may be compromised.
Each fault condition is specified as a mathematically precise inequality
or predicate.

\subsubsection{Fault Classification}\label{fault-classification}

Faults are classified into four categories:

\begin{enumerate}
\def\labelenumi{\arabic{enumi}.}
\tightlist
\item
  \textbf{Data Quality Faults}: Violations of input data validity
  assumptions
\item
  \textbf{Model Validity Faults}: Conditions where pricing models may be
  unreliable\\
\item
  \textbf{Execution Risk Faults}: Conditions threatening profitable
  execution
\item
  \textbf{Position Risk Faults}: Conditions requiring position
  adjustment or exit
\end{enumerate}

\subsection{Data Quality Validation}\label{data-quality-validation}

\subsubsection{Price Data Validation}\label{price-data-validation}

\begin{definition}[Price Data Validity
Predicates]\protect\hypertarget{def-price-validation}{}\label{def-price-validation}

For price data point \(P_i = (O_i, H_i, L_i, C_i, V_i)\):

\begin{align}
\mathcal{D}_1(P_i) &\equiv O_i > 0 \land H_i > 0 \land L_i > 0 \land C_i > 0 \tag{Positivity} \\
\mathcal{D}_2(P_i) &\equiv L_i \leq O_i \leq H_i \land L_i \leq C_i \leq H_i \tag{OHLC Consistency} \\
\mathcal{D}_3(P_i) &\equiv V_i \geq 0 \tag{Volume Non-negativity} \\
\mathcal{D}_4(P_i, P_{i-1}) &\equiv \left|\frac{O_i - C_{i-1}}{C_{i-1}}\right| < 0.50 \tag{Gap Reasonableness}
\end{align}

Aggregate price validity: \[
\mathcal{D}_{\text{price}}(P_i) \equiv \bigwedge_{j=1}^{4} \mathcal{D}_j
\]

\end{definition}

\subsubsection{Implied Volatility
Validation}\label{implied-volatility-validation}

\begin{definition}[IV Data Validity
Predicates]\protect\hypertarget{def-iv-validation}{}\label{def-iv-validation}

For implied volatility observation \(\sigma_I\):

\begin{align}
\mathcal{D}_5(\sigma_I) &\equiv 0.01 \leq \sigma_I \leq 5.00 \tag{Range Constraint} \\
\mathcal{D}_6(\sigma_I, \sigma_I') &\equiv \left|\sigma_I - \sigma_I'\right| < 0.50 \tag{Temporal Continuity}
\end{align}

\end{definition}

\subsubsection{Volume and Open Interest
Validation}\label{volume-and-open-interest-validation}

\begin{definition}[Liquidity Data
Validity]\protect\hypertarget{def-liquidity-validation}{}\label{def-liquidity-validation}

For option contract with volume \(V\) and open interest \(OI\):

\begin{align}
\mathcal{D}_7(V, OI) &\equiv V \geq 0 \land OI \geq 0 \tag{Non-negativity} \\
\mathcal{D}_8(V, OI) &\equiv V \leq 10 \cdot OI \tag{Volume/OI Ratio}
\end{align}

\end{definition}

\subsection{Execution Cost Validation}\label{execution-cost-validation}

\subsubsection{Cost Model Specification}\label{cost-model-specification}

The execution cost model computes total transaction cost as:

\begin{equation}\phantomsection\label{eq-exec-cost}{
C_{\text{exec}}(n) = C_{\text{spread}}(n) + C_{\text{commission}}(n) + C_{\text{slippage}}(n)
}\end{equation}

where \(n\) is the number of contracts.

\textbf{Spread Cost}: \[
C_{\text{spread}}(n) = \frac{(A - B)}{2} \cdot n \cdot 100
\]

where \(A\) is the ask price and \(B\) is the bid price.

\textbf{Commission Cost}: \[
C_{\text{commission}}(n) = (\phi_{\text{broker}} + \phi_{\text{exchange}} + \phi_{\text{regulatory}}) \cdot n
\]

with \(\phi_{\text{broker}} = 0.65\), \(\phi_{\text{exchange}} = 0.30\),
\(\phi_{\text{regulatory}} = 0.02\).

\subsubsection{Cost Survival Predicate}\label{cost-survival-predicate}

\begin{definition}[Execution Cost
Survival]\protect\hypertarget{def-cost-survival}{}\label{def-cost-survival}

The post-cost IV/RV ratio must remain above threshold:

\begin{equation}\phantomsection\label{eq-cost-survival}{
\mathcal{F}_{\text{cost}}(\xi) \equiv \frac{\sigma_I^{30}(\xi)}{\sigma_R^{30}(\xi)} \cdot \left(1 - \frac{C_{\text{exec}}}{V_{\text{position}}}\right) \geq 1.20
}\end{equation}

where \(V_{\text{position}}\) is the position value.

\end{definition}

\subsection{Vega Correlation Analysis}\label{vega-correlation-analysis}

\subsubsection{Independence Criterion}\label{independence-criterion}

The calendar spread strategy requires that front-month and back-month
implied volatilities exhibit low correlation, ensuring that the spread
provides genuine vega hedging.

\begin{definition}[Vega Independence
Predicate]\protect\hypertarget{def-vega-independence}{}\label{def-vega-independence}

Let \(\{\sigma_{I,i}^{(F)}\}_{i=1}^{N}\) and
\(\{\sigma_{I,i}^{(B)}\}_{i=1}^{N}\) be historical IV observations for
front and back months.

The Pearson correlation coefficient is:

\begin{equation}\phantomsection\label{eq-correlation}{
\rho_{FB} = \frac{\sum_{i=1}^{N} \left(\sigma_{I,i}^{(F)} - \overline{\sigma_I^{(F)}}\right)\left(\sigma_{I,i}^{(B)} - \overline{\sigma_I^{(B)}}\right)}{\sqrt{\sum_{i=1}^{N} \left(\sigma_{I,i}^{(F)} - \overline{\sigma_I^{(F)}}\right)^2} \cdot \sqrt{\sum_{i=1}^{N} \left(\sigma_{I,i}^{(B)} - \overline{\sigma_I^{(B)}}\right)^2}}
}\end{equation}

The vega independence fault predicate is:

\begin{equation}\phantomsection\label{eq-vega-fault}{
\mathcal{F}_{\text{vega}}(\xi) \equiv |\rho_{FB}(\xi)| > 0.70
}\end{equation}

If \(\mathcal{F}_{\text{vega}}\) is true, the trade should not proceed
due to elevated sympathetic collapse risk.

\end{definition}

\subsection{Liquidity Assurance}\label{liquidity-assurance}

\subsubsection{Position-to-Liquidity
Constraints}\label{position-to-liquidity-constraints}

\begin{definition}[Liquidity Fault
Predicates]\protect\hypertarget{def-liquidity-constraints}{}\label{def-liquidity-constraints}

For position size \(n\) contracts with daily volume \(V\) and open
interest \(OI\):

\begin{align}
\mathcal{F}_{\text{vol}}(n) &\equiv \frac{n}{V} > 0.05 \tag{Volume Ratio Fault} \\
\mathcal{F}_{\text{OI}}(n) &\equiv \frac{n}{OI} > 0.02 \tag{Open Interest Ratio Fault}
\end{align}

Aggregate liquidity fault:
\begin{equation}\phantomsection\label{eq-liquidity-fault}{
\mathcal{F}_{\text{liq}}(n) \equiv \mathcal{F}_{\text{vol}}(n) \lor \mathcal{F}_{\text{OI}}(n)
}\end{equation}

\end{definition}

\subsection{Gamma Risk Monitoring}\label{gamma-risk-monitoring}

\subsubsection{Delta-Neutral
Maintenance}\label{delta-neutral-maintenance}

The calendar spread strategy aims to maintain delta-neutrality.
Monitoring detects when delta exceeds acceptable bounds.

\begin{definition}[Delta Drift
Fault]\protect\hypertarget{def-delta-monitoring}{}\label{def-delta-monitoring}

For position with net delta \(\Delta_{\text{net}}\):

\begin{equation}\phantomsection\label{eq-delta-fault}{
\mathcal{F}_{\text{delta}}(\Delta_{\text{net}}) \equiv |\Delta_{\text{net}}| > 0.10
}\end{equation}

\end{definition}

\subsubsection{Gamma Warning Threshold}\label{gamma-warning-threshold}

\begin{definition}[Gamma Risk
Fault]\protect\hypertarget{def-gamma-monitoring}{}\label{def-gamma-monitoring}

For position with net gamma \(\Gamma_{\text{net}}\):

\begin{equation}\phantomsection\label{eq-gamma-fault}{
\mathcal{F}_{\text{gamma}}(\Gamma_{\text{net}}) \equiv \Gamma_{\text{net}} < -0.05
}\end{equation}

Negative gamma indicates the position loses value as the underlying
moves in either direction, requiring heightened monitoring.

\end{definition}

\subsubsection{Moneyness Alert}\label{moneyness-alert}

\begin{definition}[Moneyness
Fault]\protect\hypertarget{def-moneyness-alert}{}\label{def-moneyness-alert}

For underlying price \(S\) and strike \(K\):

\begin{equation}\phantomsection\label{eq-moneyness-fault}{
\mathcal{F}_{\text{moneyness}}(S, K) \equiv \left|\frac{S - K}{K}\right| > 0.03
}\end{equation}

Calendar spreads function optimally near the money; significant drift
requires re-evaluation.

\end{definition}

\subsection{Circuit Breaker
Conditions}\label{circuit-breaker-conditions}

\subsubsection{System-Level Halts}\label{system-level-halts}

\begin{definition}[Circuit Breaker
Predicates]\protect\hypertarget{def-circuit-breakers}{}\label{def-circuit-breakers}

\textbf{Position Loss Circuit Breaker}:
\begin{equation}\phantomsection\label{eq-loss-breaker}{
\mathcal{C}_{\text{loss}}(\xi) \equiv \frac{P\&L_{\text{daily}}(\xi)}{\text{Capital}} < -0.05
}\end{equation}

\textbf{Data Feed Circuit Breaker}:
\begin{equation}\phantomsection\label{eq-data-breaker}{
\mathcal{C}_{\text{data}} \equiv t_{\text{current}} - t_{\text{last update}} > 15 \text{ minutes}
}\end{equation}

\textbf{Validation Failure Circuit Breaker}:
\begin{equation}\phantomsection\label{eq-valid-breaker}{
\mathcal{C}_{\text{valid}} \equiv \frac{\#\text{validation failures}}{\#\text{total checks}} > 0.10
}\end{equation}

\textbf{Volatility Spike Circuit Breaker}:
\begin{equation}\phantomsection\label{eq-vix-breaker}{
\mathcal{C}_{\text{VIX}} \equiv \text{VIX} > 40 \lor \sigma_R(\xi) > 2 \cdot \bar{\sigma}_R(\xi)
}\end{equation}

\end{definition}

\subsection{Production Validation
Aggregation}\label{production-validation-aggregation}

\subsubsection{Composite Production
Readiness}\label{composite-production-readiness}

\begin{definition}[Production Readiness
Predicate]\protect\hypertarget{def-production-ready}{}\label{def-production-ready}

A signal is production-ready if and only if:

\begin{equation}\phantomsection\label{eq-production-ready}{
\mathcal{P}_{\text{ready}}(\xi) \equiv \Sigma(\xi) = \text{Recommended} \land \bigwedge_{i \in \mathcal{I}} \neg\mathcal{F}_i(\xi)
}\end{equation}

where
\(\mathcal{I} = \{\text{cost}, \text{vega}, \text{liq}, \text{delta}, \text{gamma}, \text{moneyness}\}\)
is the index set of all fault predicates.

\end{definition}

The production validator (Component STHD005A) implements this predicate
composition, returning a structured result indicating which conditions
passed and which failed.

\section{Stochasticity and
Non-Determinism}\label{stochasticity-and-non-determinism}

\subsection{Inherently Stochastic
Components}\label{inherently-stochastic-components}

Certain components of the Alaris system involve quantities that are
fundamentally stochastic. This section explicitly characterises these
components and the statistical methods used to handle them.

\subsubsection{Asset Price Evolution}\label{asset-price-evolution}

The underlying asset price follows a continuous-time stochastic process
(Equation~\ref{eq-sde}). \textbf{This is not a fault}---it is the
defining characteristic of the problem domain. The system handles this
through:

\begin{enumerate}
\def\labelenumi{\arabic{enumi}.}
\tightlist
\item
  \textbf{Risk-neutral pricing}: Option values are computed as
  expectations under \(\mathbb{Q}\), eliminating the need to forecast
  price direction
\item
  \textbf{Delta hedging}: The strategy is approximately delta-neutral,
  reducing sensitivity to price movements
\end{enumerate}

\subsubsection{Volatility Estimation}\label{volatility-estimation}

Realised volatility estimation from finite samples is inherently subject
to statistical uncertainty.

\phantomsection\label{prop-yz-distribution}
\subsection{Yang-Zhang Estimator
Distribution}\label{yang-zhang-estimator-distribution}

Under the assumption of geometric Brownian motion, the Yang-Zhang
variance estimator \(\hat{\sigma}_{YZ}^2\) satisfies:

\[
\frac{(n-1)\hat{\sigma}_{YZ}^2}{\sigma^2} \sim \chi^2_{n-1 + \eta(n-1)}
\]

approximately, where \(\eta \approx 7\) is the efficiency factor.

The \((1-\alpha)\) confidence interval for \(\sigma^2\) is:

\[
\left[\frac{(n-1)\hat{\sigma}_{YZ}^2}{\chi^2_{n-1, 1-\alpha/2}}, \frac{(n-1)\hat{\sigma}_{YZ}^2}{\chi^2_{n-1, \alpha/2}}\right]
\]

\textbf{Accounting for this uncertainty}: The IV/RV ratio threshold
(1.25) incorporates a buffer above 1.0 to account for estimation error
in the denominator.

\subsubsection{Implied Volatility
Surface}\label{implied-volatility-surface}

Implied volatility observations are subject to:

\begin{enumerate}
\def\labelenumi{\arabic{enumi}.}
\tightlist
\item
  \textbf{Bid-ask spread noise}: Mid-prices may not reflect true market
  clearing levels
\item
  \textbf{Temporal asynchrony}: Options may have traded at different
  times
\item
  \textbf{Model error}: Black-Scholes implied volatility assumes
  constant volatility
\end{enumerate}

\textbf{Accounting for these}: The term structure analysis uses robust
regression on multiple strike/expiry points, and the slope threshold
(-0.00406) is calibrated from empirical studies.

\subsection{Deterministic Components}\label{deterministic-components}

The following components are fully deterministic given their inputs:

\subsubsection{Boundary Computation}\label{boundary-computation}

Given parameters \((S, K, T, r, q, \sigma)\), the computed boundaries
\(B(\tau)\) and \(Y(\tau)\) are deterministic. The only source of
variation is:

\begin{enumerate}
\def\labelenumi{\arabic{enumi}.}
\tightlist
\item
  \textbf{Numerical precision}: Controlled by specified tolerances
  (\(\epsilon = 10^{-10}\))
\item
  \textbf{Iteration limits}: Maximum iterations prevent infinite loops
\end{enumerate}

\subsubsection{Signal Generation}\label{signal-generation}

Given input data, the signal strength \(\Sigma\) is deterministic:

\[
\Sigma = \Sigma(\mathcal{S}_1, \mathcal{S}_2, \mathcal{S}_3)
\]

Each \(\mathcal{S}_i\) is a deterministic function of the input data.

\subsubsection{Fault Detection}\label{fault-detection}

All fault predicates \(\mathcal{F}_i\) are deterministic Boolean
functions of system state.

\subsection{Accounting Summary}\label{accounting-summary}

\begin{longtable}[]{@{}
  >{\raggedright\arraybackslash}p{(\linewidth - 4\tabcolsep) * \real{0.2558}}
  >{\raggedright\arraybackslash}p{(\linewidth - 4\tabcolsep) * \real{0.3023}}
  >{\raggedright\arraybackslash}p{(\linewidth - 4\tabcolsep) * \real{0.4419}}@{}}
\caption{Component Stochasticity
Summary}\label{tbl-stochasticity}\tabularnewline
\toprule\noalign{}
\begin{minipage}[b]{\linewidth}\raggedright
Component
\end{minipage} & \begin{minipage}[b]{\linewidth}\raggedright
Stochastic?
\end{minipage} & \begin{minipage}[b]{\linewidth}\raggedright
Accounting Method
\end{minipage} \\
\midrule\noalign{}
\endfirsthead
\toprule\noalign{}
\begin{minipage}[b]{\linewidth}\raggedright
Component
\end{minipage} & \begin{minipage}[b]{\linewidth}\raggedright
Stochastic?
\end{minipage} & \begin{minipage}[b]{\linewidth}\raggedright
Accounting Method
\end{minipage} \\
\midrule\noalign{}
\endhead
\bottomrule\noalign{}
\endlastfoot
Asset price & Yes & Risk-neutral pricing, delta hedging \\
Realised volatility & Yes (estimation) & Yang-Zhang estimator, ratio
buffer \\
Implied volatility & Yes (measurement) & Multiple observations, robust
fitting \\
Boundary computation & No & Tolerance specification \\
Signal generation & No & Deterministic predicate evaluation \\
Fault detection & No & Deterministic inequality evaluation \\
Position sizing & No & Deterministic Kelly formula \\
\end{longtable}

\section{Numerical Methods and
Implementation}\label{numerical-methods-and-implementation}

\subsection{Collocation Method for Boundary
Refinement}\label{collocation-method-for-boundary-refinement}

\subsubsection{Chebyshev Node
Distribution}\label{chebyshev-node-distribution}

The Kim integral equation is discretised using Chebyshev collocation
points for optimal polynomial interpolation:

\begin{equation}\phantomsection\label{eq-chebyshev}{
\xi_j = \cos\left(\frac{(2j-1)\pi}{2N}\right), \quad j = 1, \ldots, N
}\end{equation}

These nodes are mapped to the time domain \([0, T]\) via:

\[
\tau_j = \frac{T}{2}(1 + \xi_j)
\]

\subsubsection{Spectral Convergence}\label{spectral-convergence}

\begin{theorem}[Spectral
Convergence]\protect\hypertarget{thm-spectral}{}\label{thm-spectral}

For analytic integrands, the collocation method achieves exponential
convergence:

\[
\|B_N - B_{\text{exact}}\|_{\infty} = O(e^{-cN})
\]

for some constant \(c > 0\) depending on the analyticity domain.

\end{theorem}

The Alaris system uses \(N = 50\) collocation points, providing accuracy
exceeding the Healy \citeproc{ref-healy2021}{(2021)} benchmark values.

\subsection{Convergence Criteria}\label{convergence-criteria}

\subsubsection{Iteration Termination}\label{iteration-termination}

The FP-B' iteration terminates when:

\begin{equation}\phantomsection\label{eq-convergence}{
\max_{j=1,\ldots,N} |B_j^{(n+1)} - B_j^{(n)}| < \epsilon_{\text{tol}}
}\end{equation}

with \(\epsilon_{\text{tol}} = 10^{-6}\) and maximum iterations
\(N_{\max} = 100\).

\subsection{Benchmark Validation}\label{benchmark-validation}

The implementation is validated against Healy
\citeproc{ref-healy2021}{(2021)} Table 2 benchmark values:

\begin{longtable}[]{@{}
  >{\raggedright\arraybackslash}p{(\linewidth - 10\tabcolsep) * \real{0.1275}}
  >{\raggedright\arraybackslash}p{(\linewidth - 10\tabcolsep) * \real{0.1961}}
  >{\raggedright\arraybackslash}p{(\linewidth - 10\tabcolsep) * \real{0.1961}}
  >{\raggedright\arraybackslash}p{(\linewidth - 10\tabcolsep) * \real{0.2059}}
  >{\raggedright\arraybackslash}p{(\linewidth - 10\tabcolsep) * \real{0.2059}}
  >{\raggedright\arraybackslash}p{(\linewidth - 10\tabcolsep) * \real{0.0686}}@{}}
\caption{Healy Benchmark
Validation}\label{tbl-healy-benchmark}\tabularnewline
\toprule\noalign{}
\begin{minipage}[b]{\linewidth}\raggedright
\(T\) (years)
\end{minipage} & \begin{minipage}[b]{\linewidth}\raggedright
\(B_{\text{Healy}}\)
\end{minipage} & \begin{minipage}[b]{\linewidth}\raggedright
\(Y_{\text{Healy}}\)
\end{minipage} & \begin{minipage}[b]{\linewidth}\raggedright
\(B_{\text{Alaris}}\)
\end{minipage} & \begin{minipage}[b]{\linewidth}\raggedright
\(Y_{\text{Alaris}}\)
\end{minipage} & \begin{minipage}[b]{\linewidth}\raggedright
Error
\end{minipage} \\
\midrule\noalign{}
\endfirsthead
\toprule\noalign{}
\begin{minipage}[b]{\linewidth}\raggedright
\(T\) (years)
\end{minipage} & \begin{minipage}[b]{\linewidth}\raggedright
\(B_{\text{Healy}}\)
\end{minipage} & \begin{minipage}[b]{\linewidth}\raggedright
\(Y_{\text{Healy}}\)
\end{minipage} & \begin{minipage}[b]{\linewidth}\raggedright
\(B_{\text{Alaris}}\)
\end{minipage} & \begin{minipage}[b]{\linewidth}\raggedright
\(Y_{\text{Alaris}}\)
\end{minipage} & \begin{minipage}[b]{\linewidth}\raggedright
Error
\end{minipage} \\
\midrule\noalign{}
\endhead
\bottomrule\noalign{}
\endlastfoot
5 & 71.60 & 61.60 & 71.60 & 61.60 & \textless0.01\% \\
10 & 69.62 & 58.72 & 69.62 & 58.72 & \textless0.01\% \\
15 & 68.00 & 57.00 & 68.00 & 57.00 & \textless0.01\% \\
\end{longtable}

Parameters: \(S = K = 100\), \(r = -0.005\), \(q = -0.01\),
\(\sigma = 0.08\).

\section{Conclusion}\label{conclusion}

This document has provided a complete mathematical specification of the
Alaris quantitative trading system. The key contributions are:

\begin{enumerate}
\def\labelenumi{\arabic{enumi}.}
\item
  \textbf{Rigorous option pricing framework} extending classical methods
  to negative interest rate environments with double boundary topologies
\item
  \textbf{Formal specification of trading signals} with each criterion
  expressed as an evaluable mathematical predicate
\item
  \textbf{Complete fault detection system} specified as a collection of
  inequalities and logical predicates, ensuring deterministic monitoring
  behaviour
\item
  \textbf{Explicit treatment of stochasticity} distinguishing between
  inherently random quantities and measurement/estimation uncertainty,
  with appropriate statistical methods for each
\end{enumerate}

The mathematical framework ensures that every algorithmic decision in
the Alaris system corresponds to the evaluation of a well-defined
mathematical expression. Where randomness is inherent to the underlying
phenomena, we have specified the probabilistic models and derived
estimators with known statistical properties. Where measurements are
subject to uncertainty, we have incorporated appropriate buffers and
validation checks.

This specification serves as the authoritative reference for the Alaris
implementation and provides the foundation for formal verification of
system behaviour.

\section{References}\label{references}

\phantomsection\label{refs}
\begin{CSLReferences}{0}{1}
\bibitem[\citeproctext]{ref-andersen2016}
Andersen, Leif B. G., Mark Lake, and Dimitri Offengenden, 2016,
High-performance {A}merican option pricing, \emph{Journal of
Computational Finance} 20, 39--87.

\bibitem[\citeproctext]{ref-atilgan2014}
Atilgan, Yigit, Turan G Bali, K Ozgur Demirtas, and A Doruk Gunaydin,
2014, Implied volatility spreads and expected market returns,
\emph{Journal of Business \& Economic Statistics} 32, 212--227.

\bibitem[\citeproctext]{ref-healy2021}
Healy, Jerome V, 2021, Pricing {A}merican options under negative rates,
\emph{Journal of Derivatives} 28, 33--52.

\bibitem[\citeproctext]{ref-yang2000}
Yang, Dennis, and Qiang Zhang, 2000, Drift-independent volatility
estimation based on high, low, open, and close prices, \emph{Journal of
Business} 73, 477--492.

\end{CSLReferences}

\section{Appendix: Symbol Reference}\label{appendix-symbol-reference}

\subsection{Greek Letters}\label{greek-letters}

\begin{longtable}[]{@{}ll@{}}
\caption{Greek Letter Notation}\label{tbl-greek}\tabularnewline
\toprule\noalign{}
Symbol & Description \\
\midrule\noalign{}
\endfirsthead
\toprule\noalign{}
Symbol & Description \\
\midrule\noalign{}
\endhead
\bottomrule\noalign{}
\endlastfoot
\(\alpha\) & Term structure intercept \\
\(\beta\) & Term structure slope \\
\(\gamma, \Gamma\) & Option gamma (second derivative w.r.t. spot) \\
\(\delta, \Delta\) & Option delta (first derivative w.r.t. spot) \\
\(\epsilon\) & Error term / tolerance \\
\(\eta\) & Efficiency factor \\
\(\theta, \Theta\) & Option theta (time decay) \\
\(\kappa\) & Kelly fraction multiplier \\
\(\lambda\) & Characteristic equation roots \\
\(\mu\) & Expected return \\
\(\rho\) & Correlation coefficient \\
\(\sigma\) & Volatility \\
\(\tau\) & Time to maturity \\
\(\phi\) & Commission component \\
\(\Phi\) & Standard normal CDF \\
\(\Sigma\) & Signal strength function \\
\end{longtable}

\subsection{Roman Letters}\label{roman-letters}

\begin{longtable}[]{@{}ll@{}}
\caption{Roman Letter Notation}\label{tbl-roman}\tabularnewline
\toprule\noalign{}
Symbol & Description \\
\midrule\noalign{}
\endfirsthead
\toprule\noalign{}
Symbol & Description \\
\midrule\noalign{}
\endhead
\bottomrule\noalign{}
\endlastfoot
\(B(\tau)\) & Upper exercise boundary \\
\(Y(\tau)\) & Lower exercise boundary \\
\(C\) & Cost component \\
\(K\) & Strike price \\
\(N\) & Number of collocation points \\
\(O, H, L, C\) & Open, High, Low, Close prices \\
\(r\) & Risk-free interest rate \\
\(q\) & Dividend yield \\
\(S\) & Spot price \\
\(T\) & Maturity \\
\(V\) & Option value / Volume \\
\(v\) & European option value \\
\end{longtable}

\subsection{Calligraphic and Blackboard
Bold}\label{calligraphic-and-blackboard-bold}

\begin{longtable}[]{@{}ll@{}}
\caption{Special Notation}\label{tbl-special}\tabularnewline
\toprule\noalign{}
Symbol & Description \\
\midrule\noalign{}
\endfirsthead
\toprule\noalign{}
Symbol & Description \\
\midrule\noalign{}
\endhead
\bottomrule\noalign{}
\endlastfoot
\(\mathcal{C}\) & Circuit breaker predicate \\
\(\mathcal{D}\) & Data validation predicate \\
\(\mathcal{E}\) & Exercise region \\
\(\mathcal{F}\) & Fault predicate / Filtration \\
\(\mathcal{K}\) & Integral kernel \\
\(\mathcal{L}\) & Infinitesimal generator \\
\(\mathcal{P}\) & Early exercise premium / Production predicate \\
\(\mathcal{R}\) & Regime indicator \\
\(\mathcal{S}\) & Signal criterion predicate \\
\(\mathcal{T}\) & Set of stopping times \\
\(\mathcal{V}\) & Validity predicate \\
\(\mathbb{E}\) & Expectation operator \\
\(\mathbb{P}\) & Physical probability measure \\
\(\mathbb{Q}\) & Risk-neutral probability measure \\
\(\mathbb{R}\) & Real numbers \\
\end{longtable}




\end{document}
